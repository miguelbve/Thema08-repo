% Options for packages loaded elsewhere
\PassOptionsToPackage{unicode}{hyperref}
\PassOptionsToPackage{hyphens}{url}
\PassOptionsToPackage{dvipsnames,svgnames*,x11names*}{xcolor}
%
\documentclass[
]{article}
\usepackage{lmodern}
\usepackage{amsmath}
\usepackage{ifxetex,ifluatex}
\ifnum 0\ifxetex 1\fi\ifluatex 1\fi=0 % if pdftex
  \usepackage[T1]{fontenc}
  \usepackage[utf8]{inputenc}
  \usepackage{textcomp} % provide euro and other symbols
  \usepackage{amssymb}
\else % if luatex or xetex
  \usepackage{unicode-math}
  \defaultfontfeatures{Scale=MatchLowercase}
  \defaultfontfeatures[\rmfamily]{Ligatures=TeX,Scale=1}
\fi
% Use upquote if available, for straight quotes in verbatim environments
\IfFileExists{upquote.sty}{\usepackage{upquote}}{}
\IfFileExists{microtype.sty}{% use microtype if available
  \usepackage[]{microtype}
  \UseMicrotypeSet[protrusion]{basicmath} % disable protrusion for tt fonts
}{}
\makeatletter
\@ifundefined{KOMAClassName}{% if non-KOMA class
  \IfFileExists{parskip.sty}{%
    \usepackage{parskip}
  }{% else
    \setlength{\parindent}{0pt}
    \setlength{\parskip}{6pt plus 2pt minus 1pt}}
}{% if KOMA class
  \KOMAoptions{parskip=half}}
\makeatother
\usepackage{xcolor}
\IfFileExists{xurl.sty}{\usepackage{xurl}}{} % add URL line breaks if available
\IfFileExists{bookmark.sty}{\usepackage{bookmark}}{\usepackage{hyperref}}
\hypersetup{
  pdftitle={titel},
  pdfauthor={my\_name},
  colorlinks=true,
  linkcolor=blue,
  filecolor=Maroon,
  citecolor=Blue,
  urlcolor=Blue,
  pdfcreator={LaTeX via pandoc}}
\urlstyle{same} % disable monospaced font for URLs
\usepackage[margin=1in]{geometry}
\usepackage{color}
\usepackage{fancyvrb}
\newcommand{\VerbBar}{|}
\newcommand{\VERB}{\Verb[commandchars=\\\{\}]}
\DefineVerbatimEnvironment{Highlighting}{Verbatim}{commandchars=\\\{\}}
% Add ',fontsize=\small' for more characters per line
\usepackage{framed}
\definecolor{shadecolor}{RGB}{248,248,248}
\newenvironment{Shaded}{\begin{snugshade}}{\end{snugshade}}
\newcommand{\AlertTok}[1]{\textcolor[rgb]{0.94,0.16,0.16}{#1}}
\newcommand{\AnnotationTok}[1]{\textcolor[rgb]{0.56,0.35,0.01}{\textbf{\textit{#1}}}}
\newcommand{\AttributeTok}[1]{\textcolor[rgb]{0.77,0.63,0.00}{#1}}
\newcommand{\BaseNTok}[1]{\textcolor[rgb]{0.00,0.00,0.81}{#1}}
\newcommand{\BuiltInTok}[1]{#1}
\newcommand{\CharTok}[1]{\textcolor[rgb]{0.31,0.60,0.02}{#1}}
\newcommand{\CommentTok}[1]{\textcolor[rgb]{0.56,0.35,0.01}{\textit{#1}}}
\newcommand{\CommentVarTok}[1]{\textcolor[rgb]{0.56,0.35,0.01}{\textbf{\textit{#1}}}}
\newcommand{\ConstantTok}[1]{\textcolor[rgb]{0.00,0.00,0.00}{#1}}
\newcommand{\ControlFlowTok}[1]{\textcolor[rgb]{0.13,0.29,0.53}{\textbf{#1}}}
\newcommand{\DataTypeTok}[1]{\textcolor[rgb]{0.13,0.29,0.53}{#1}}
\newcommand{\DecValTok}[1]{\textcolor[rgb]{0.00,0.00,0.81}{#1}}
\newcommand{\DocumentationTok}[1]{\textcolor[rgb]{0.56,0.35,0.01}{\textbf{\textit{#1}}}}
\newcommand{\ErrorTok}[1]{\textcolor[rgb]{0.64,0.00,0.00}{\textbf{#1}}}
\newcommand{\ExtensionTok}[1]{#1}
\newcommand{\FloatTok}[1]{\textcolor[rgb]{0.00,0.00,0.81}{#1}}
\newcommand{\FunctionTok}[1]{\textcolor[rgb]{0.00,0.00,0.00}{#1}}
\newcommand{\ImportTok}[1]{#1}
\newcommand{\InformationTok}[1]{\textcolor[rgb]{0.56,0.35,0.01}{\textbf{\textit{#1}}}}
\newcommand{\KeywordTok}[1]{\textcolor[rgb]{0.13,0.29,0.53}{\textbf{#1}}}
\newcommand{\NormalTok}[1]{#1}
\newcommand{\OperatorTok}[1]{\textcolor[rgb]{0.81,0.36,0.00}{\textbf{#1}}}
\newcommand{\OtherTok}[1]{\textcolor[rgb]{0.56,0.35,0.01}{#1}}
\newcommand{\PreprocessorTok}[1]{\textcolor[rgb]{0.56,0.35,0.01}{\textit{#1}}}
\newcommand{\RegionMarkerTok}[1]{#1}
\newcommand{\SpecialCharTok}[1]{\textcolor[rgb]{0.00,0.00,0.00}{#1}}
\newcommand{\SpecialStringTok}[1]{\textcolor[rgb]{0.31,0.60,0.02}{#1}}
\newcommand{\StringTok}[1]{\textcolor[rgb]{0.31,0.60,0.02}{#1}}
\newcommand{\VariableTok}[1]{\textcolor[rgb]{0.00,0.00,0.00}{#1}}
\newcommand{\VerbatimStringTok}[1]{\textcolor[rgb]{0.31,0.60,0.02}{#1}}
\newcommand{\WarningTok}[1]{\textcolor[rgb]{0.56,0.35,0.01}{\textbf{\textit{#1}}}}
\usepackage{graphicx}
\makeatletter
\def\maxwidth{\ifdim\Gin@nat@width>\linewidth\linewidth\else\Gin@nat@width\fi}
\def\maxheight{\ifdim\Gin@nat@height>\textheight\textheight\else\Gin@nat@height\fi}
\makeatother
% Scale images if necessary, so that they will not overflow the page
% margins by default, and it is still possible to overwrite the defaults
% using explicit options in \includegraphics[width, height, ...]{}
\setkeys{Gin}{width=\maxwidth,height=\maxheight,keepaspectratio}
% Set default figure placement to htbp
\makeatletter
\def\fps@figure{htbp}
\makeatother
\setlength{\emergencystretch}{3em} % prevent overfull lines
\providecommand{\tightlist}{%
  \setlength{\itemsep}{0pt}\setlength{\parskip}{0pt}}
\setcounter{secnumdepth}{5}
\usepackage{longtable}
\usepackage{hyperref}
\ifluatex
  \usepackage{selnolig}  % disable illegal ligatures
\fi

\title{titel}
\author{my\_name}
\date{2021-06-02}

\begin{document}
\maketitle

\hypertarget{introduction}{%
\section{Introduction}\label{introduction}}

Introduction of the research and introduction research questions

The effect of plasma concentration should be related to the
concentration of the test substance (so it implies the delayed
ventricular regulation). The concentration of the test substance is
highly effected by the extent of the delayed ventricular repolarization.
Since the plasma concentration is most commonly used as the effective
concentration. This research is interested in the mean concentration
from drug where it meet cardiac ion channels within heart tissue.

Drug concentration in heart tissue should be of particular interest
regarding all possible sites where the drug might meet cardiac ion
channels.

\hypertarget{goal}{%
\subsection{Goal}\label{goal}}

\begin{itemize}
\tightlist
\item
  Describe Goal (not the educational goal but the research goal)
\item
  Describe how you reach the goal (e.g.~make model and figures, use
  different setting)
\item
  formulate hypothesis
\end{itemize}

This research aims to give a good understanding of the drug
concentration over time in different tissue types of the heart. A PBPK
(physiologically-based pharmacokinetic) approach has hardly been used in
the modeling of drug concentration in various locations within heart
tissue so this might be a great opportunity to accelerate research into
the role of drugs on many things heart related. In order to get a better
understanding of this research, deSolve will be issued and model
configurations should be set. In addition defined equatuions will be
solved to enlarge the knowledge about the drug concentration. Therefore
the Null-hypothesis; there is no significant difference between
different sample groups, will be tested. If that's not the case then the
alternative hypothesis is assumed to be true.

\hypertarget{theory}{%
\subsection{Theory}\label{theory}}

In order to understand the model, the concept of a PBPK model needs to
be clear. Simpy put, PBPK is a computer modeling approach that
incorporates blood flow and tissue composition of organs to define the
pharmacokinetics (PK) of drugs. Alterations in PK properties, such as,
absorption, distribution, metabolism, and excretion, can have a
substantial impact on achieving the desired therapeutic concentration of
a drug. PBPK is a very powerful tool, so a lot of computing power is
necessary. The best use for a PBPK model is drug research, which is the
reason this type of model will be used in this research. It can also be,
for instance, used by the Pharmaceutical Industry. The essention of
integral calculations results in surface area of the graph, which is a
useful application for dynamic models like the PBPK model.

The equations which are part of the heart PBPK model are as
follows:(UITLEG EQUATIONS)

\begin{enumerate}
\def\labelenumi{(\arabic{enumi})}
\item
  \begin{itemize}
  \tightlist
  \item
    Vmax{[}mg/h{]} = Vmax\_pmol x CYP x MPPGL x Wli x MW x 60 / 10\^{}9
    \[{V{_m}}{_a}{_x}{__}{_p}{_m}{_o}{BP}*CLu{_i}\]
  \end{itemize}
\item
  \begin{itemize}
  \tightlist
  \item
    CLu int2C8 = Cl int2C8 x fumic x ISEF2C8
  \end{itemize}
\item
  \begin{itemize}
  \tightlist
  \item
    \[\frac{\frac{fu{_p}}{BP}*CLu{_i}{_n}{_t}*Q{_h}{_e}}{Q{_h}{_e}+{\frac{fu{_p}}{BP}*CLu{_i}{_n}{_t}}}\]
  \end{itemize}
\end{enumerate}

\ldots{} This part is work in progress\ldots{}

\hypertarget{methods}{%
\section{Methods}\label{methods}}

\hypertarget{the-software-model}{%
\subsection{The software model}\label{the-software-model}}

The tool used for this experiment is called deSolve. This is a R-package
which can help solve ODE, or ordinary differential equations. A few
parameters were gotten from another research which used a program
called: ``Simcyp Simulator'' to create a PBPK model. This model can
predict certain values for tissues like the Kp which is pretty useful in
this case. The research talked about just now has also used a package
called FME, which performs a model fit based on algorithms. Equation 4
also plays a crucial role in this step.

\ldots{} This part is work in progress\ldots{}

\begin{Shaded}
\begin{Highlighting}[]
\FunctionTok{library}\NormalTok{(deSolve)}
\CommentTok{\# code}
\end{Highlighting}
\end{Shaded}

\hypertarget{model-configuration}{%
\subsection{Model configuration}\label{model-configuration}}

Chosen parameter- and initial values can be found in the tables below.
Do please note that each table corresponds to their own model
respectively.

\ldots{} This part is work in progress\ldots{}

\hypertarget{results}{%
\section{Results}\label{results}}

Introduction of results, how does it answer your research questions.

{[}MIGUEL{]}

\begin{Shaded}
\begin{Highlighting}[]
\CommentTok{\#plot(out)}
\CommentTok{\#code to generate figures with title, subscripts, legenda etc}
\end{Highlighting}
\end{Shaded}

\begin{itemize}
\tightlist
\item
  Describe what can be seen in such way that it leads to an answer to
  your research questions
\item
  Give your figures a number and a descriptive title.
\item
  Provide correct axis labels (unit and quantity), legend and caption.
\item
  Always refer to and discuss your figures and tables in the text - they
  never stand alone.
\end{itemize}

\hypertarget{discussion-and-conclusion}{%
\section{Discussion and Conclusion}\label{discussion-and-conclusion}}

\hypertarget{discussion}{%
\subsection{Discussion}\label{discussion}}

\begin{itemize}
\tightlist
\item
  Compare your results with what is expecting from the literature and
  discuss differences with them.
\item
  Discuss striking and surprising results.
\item
  Discuss weaknesses in your research and how they could be addressed.
\end{itemize}

\hypertarget{general-conclusion-and-perspective}{%
\subsection{General conclusion and
perspective}\label{general-conclusion-and-perspective}}

Discuss what your goal was, what the end result is and how you could
continue working from here.

\begin{thebibliography}{9}

\bibitem{Soertaert10}
Soetaert, K., Petzoldt, T., and Woodrow Setzer, R.: \textit{Solving differential equations in R: package deSolve}, J. Stat. Softw., 33, 1-25, 2010.

\end{thebibliography}

\end{document}
